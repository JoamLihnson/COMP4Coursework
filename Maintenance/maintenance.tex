\chapter{System Maintenance}

\section{Environment}

\subsection{Software}
Throughout the development of my program, I have used a wide variety of different programs\begin{itemize}
    \item Python 3.4.1
    \item Python's IDLE (Integrated DeveLopment Environment)
    \item PyQt 4
    \item SQLite3
    \item SQLite Inspector
    \item Mozilla Firefox
\end{itemize}


\subsection{Usage Explanation}

\begin{center}
    \begin{tabular}{|p{3cm}|p{8cm}|}
	\hline
	\textbf{Software} & \textbf{Explanation} \\ \hline
	{Python 3.4.1} & {Python is my most proficient 4th generation language to create with. It is the most ideal programming language I know currently for the problems I wanted to solve. Version 3.4.1 is also ideal as some very useful modules only work with this version type.} \\ \hline
	{Python's IDLE} & {This was used for typing out the code. It provides very useful code-coloring for syntax so its simple to see if you've made certain errors before compiling. It also gives options for character editing so you can make it easier to see your code and in a preferred font.} \\ \hline
	{PyQt 4} & {This is the GUI module I'm using to create the visual part of the software. It works well to achieve the kind of visual look and abilities I want in my program.} \\ \hline
	{SQLite3} & {This is the database module I'm using to create and manage the database part of the program. It allows me to use SQL to manage database files from Python, meaning it can be inbuilt into the program.} \\ \hline
	{SQLite Inspector} & {A database viewing software. This was used for seeing if my Python code was successful in interacting with a database and testing SQL statements before adding them to my program.} \\ \hline
	{Mozilla Firefox} & {A web browser I used to download the various programs and modules needed in the creation of my software, as well as letting solve issues with my program by searching the issue on line.} \\ \hline

    \end{tabular}
\end{center}

\section{System Overview}

%use as many subsections as necessary for the system components
\subsection{System Component}

\section{Code Structure}

%use as many subsections as necessary for the code sections
\subsection{Particular Code Section}
%the code below can be uncommented and used to get a code section from a particular file
\begin{comment}
\begin{figure}[H]
    \pythonfile[firstline=5,lastline=10]{./tex/function_programs/print_function.py}
    \caption{The print() function} \label{fig:print_function}
\end{figure}
\end{comment}

\section{Variable Listing}

\section{System Evidence}

\subsection{User Interface}

\subsection{ER Diagram}

\subsection{Database Table Views}

\subsection{Database SQL}

\subsection{SQL Queries}

\section{Testing}

My test results show that a clear indication of an incomplete project. The vast majority of the planned functionality is missing, even to go as far as a lack of interaction with a database file. Admittedly, due to the lack of functionality in the program, it did make the testing section rather easy to fill out. I guess for a solution to my issues would be to complete more of the functionality of the program, even if it was buggy or unstable.
\subsection{Summary of Results}

\subsection{Known Issues}

\section{Code Explanations}

\subsection{Difficult Sections}

\subsection{Self-created Algorithms}

\section{Settings}

\section{Acknowledgements}

\section{Code Listing}
\begin{landscape}
%include as many subsections as you have modules
\subsection{Module 1}
%the code below can be uncommented and used to get a code section from a particular file
\begin{comment}
\pythonfile[firstline=5]{./tex/function_programs/print_function.py}
\end{comment}
\end{landscape}
