\chapter{User Manual}

\section{Introduction}
The purpose of my system is to manage a database for a charity shop, being able to manage the items coming in, the items going out and the staff that work at the show. The intended demographic in this instance is the owner of the Hertfordshire based charity shop "Rusty Scraps", Judy Rust. The program uses a tandem of a GUI (Graphical User Interface) and a CLI (Command Line Interface) to give users access to the methods of adding donations and selling items, all while backing all of it up.

Unfortunately, the functionality of the program is severely lacking, with no actual ability to interact with the database file in any meaningful way. For the intended audience, I personally would recommend using whatever system is currently in place for the foreseeable future, as the program would not function as a appropriate alternative.

\section{Installation}

\subsection{Prerequisite Installation}

%WRITE HERE ALL THE HARDWARE AND SOFTWARE REQUIREMENTS
This piece of software has been built and tested on the operating system Windows 7. It works on both the 32bit and 64bit versions, so as long as you are running Windows 7, the program will be compatible. The software has not been tested on any other operating system, but it is assumed that it should work on every version of Windows from the release of Windows XP onward.

The total size of the program and the packages it comes with is about 40MB. Therefore, you require at least 40MB of space to install and use this software. You can check if you have enough space by following the instructions below:
\begin{enumerate}
    \item Click on the Start Menu (the 'Windows' button in the bottom left of the screen)
    \item Locate and click on the the 'Computer' button
    \item Depending on your sys
    \item 
    \item 
    \item 

\end{enumerate}

%include as many subsubsections as necessary for each piece of required software

\subsubsection{Program Download}
This is where you download the installer for the main program. Please make sure you have your computer connected to the internet before attempting any of this. If you are using a modern Windows operating system (Windows XP, Windows Vista, Windows 7 or Windows 8), you can see if your system is connected to the internet by looking at the right of the task bar (the long bar at the bottom of your screen). If you can see either of the below icons on your taskbar, you are connected to the internet.

\begin{figure}[H]
    \includegraphics[width=\textwidth]{./Manual/Images/Connection.png}
\end{figure}
 If not, you need to refer to separate guidance on how to obtain an internet connection.

With your working connection, you can begin to download the installer for the program. Please follow the instructions below:

\subsubsection{Installing PyQt}

\subsubsection{Etc.}

\subsection{System Installation}
db.tt/gQ3s4ArR
\subsection{Running the System}

\section{Tutorial}

\subsection{Introduction}

\subsection{Assumptions}
In an above section, I have assumed that should someone who has very sparse knowledge on internet browsers, that if somebody has taken the time to install and use a browser on a Windows operating system that isn't Internet Explorer (as it comes pre-installed with the mentioned browser), then they would be able to download the installer without a tutorial that uses pictures and annotations. This is because if you install a different internet browser to the pre-installed one, they must have experience on downloading files using an internet browser before, and thus do not require guidance on how to download and run an installer.

\subsection{Tutorial Questions}

%include as many subsubsections as necessary for each question in your list
\subsubsection{Question 1}

\subsubsection{Question 2}

\subsection{Saving}

\subsection{Limitations}

\section{Error Recovery}

%include as many subsections as necessary for each error
\subsection{Error 1}

\subsection{Error 2}

\section{System Recovery}

\subsection{Backing-up Data}

\subsection{Restoring Data}
